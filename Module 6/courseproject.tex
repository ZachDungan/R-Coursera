% Options for packages loaded elsewhere
\PassOptionsToPackage{unicode}{hyperref}
\PassOptionsToPackage{hyphens}{url}
%
\documentclass[
]{article}
\title{Course Project}
\author{Zach Dungan}
\date{3/1/2022}

\usepackage{amsmath,amssymb}
\usepackage{lmodern}
\usepackage{iftex}
\ifPDFTeX
  \usepackage[T1]{fontenc}
  \usepackage[utf8]{inputenc}
  \usepackage{textcomp} % provide euro and other symbols
\else % if luatex or xetex
  \usepackage{unicode-math}
  \defaultfontfeatures{Scale=MatchLowercase}
  \defaultfontfeatures[\rmfamily]{Ligatures=TeX,Scale=1}
\fi
% Use upquote if available, for straight quotes in verbatim environments
\IfFileExists{upquote.sty}{\usepackage{upquote}}{}
\IfFileExists{microtype.sty}{% use microtype if available
  \usepackage[]{microtype}
  \UseMicrotypeSet[protrusion]{basicmath} % disable protrusion for tt fonts
}{}
\makeatletter
\@ifundefined{KOMAClassName}{% if non-KOMA class
  \IfFileExists{parskip.sty}{%
    \usepackage{parskip}
  }{% else
    \setlength{\parindent}{0pt}
    \setlength{\parskip}{6pt plus 2pt minus 1pt}}
}{% if KOMA class
  \KOMAoptions{parskip=half}}
\makeatother
\usepackage{xcolor}
\IfFileExists{xurl.sty}{\usepackage{xurl}}{} % add URL line breaks if available
\IfFileExists{bookmark.sty}{\usepackage{bookmark}}{\usepackage{hyperref}}
\hypersetup{
  pdftitle={Course Project},
  pdfauthor={Zach Dungan},
  hidelinks,
  pdfcreator={LaTeX via pandoc}}
\urlstyle{same} % disable monospaced font for URLs
\usepackage[margin=1in]{geometry}
\usepackage{color}
\usepackage{fancyvrb}
\newcommand{\VerbBar}{|}
\newcommand{\VERB}{\Verb[commandchars=\\\{\}]}
\DefineVerbatimEnvironment{Highlighting}{Verbatim}{commandchars=\\\{\}}
% Add ',fontsize=\small' for more characters per line
\usepackage{framed}
\definecolor{shadecolor}{RGB}{248,248,248}
\newenvironment{Shaded}{\begin{snugshade}}{\end{snugshade}}
\newcommand{\AlertTok}[1]{\textcolor[rgb]{0.94,0.16,0.16}{#1}}
\newcommand{\AnnotationTok}[1]{\textcolor[rgb]{0.56,0.35,0.01}{\textbf{\textit{#1}}}}
\newcommand{\AttributeTok}[1]{\textcolor[rgb]{0.77,0.63,0.00}{#1}}
\newcommand{\BaseNTok}[1]{\textcolor[rgb]{0.00,0.00,0.81}{#1}}
\newcommand{\BuiltInTok}[1]{#1}
\newcommand{\CharTok}[1]{\textcolor[rgb]{0.31,0.60,0.02}{#1}}
\newcommand{\CommentTok}[1]{\textcolor[rgb]{0.56,0.35,0.01}{\textit{#1}}}
\newcommand{\CommentVarTok}[1]{\textcolor[rgb]{0.56,0.35,0.01}{\textbf{\textit{#1}}}}
\newcommand{\ConstantTok}[1]{\textcolor[rgb]{0.00,0.00,0.00}{#1}}
\newcommand{\ControlFlowTok}[1]{\textcolor[rgb]{0.13,0.29,0.53}{\textbf{#1}}}
\newcommand{\DataTypeTok}[1]{\textcolor[rgb]{0.13,0.29,0.53}{#1}}
\newcommand{\DecValTok}[1]{\textcolor[rgb]{0.00,0.00,0.81}{#1}}
\newcommand{\DocumentationTok}[1]{\textcolor[rgb]{0.56,0.35,0.01}{\textbf{\textit{#1}}}}
\newcommand{\ErrorTok}[1]{\textcolor[rgb]{0.64,0.00,0.00}{\textbf{#1}}}
\newcommand{\ExtensionTok}[1]{#1}
\newcommand{\FloatTok}[1]{\textcolor[rgb]{0.00,0.00,0.81}{#1}}
\newcommand{\FunctionTok}[1]{\textcolor[rgb]{0.00,0.00,0.00}{#1}}
\newcommand{\ImportTok}[1]{#1}
\newcommand{\InformationTok}[1]{\textcolor[rgb]{0.56,0.35,0.01}{\textbf{\textit{#1}}}}
\newcommand{\KeywordTok}[1]{\textcolor[rgb]{0.13,0.29,0.53}{\textbf{#1}}}
\newcommand{\NormalTok}[1]{#1}
\newcommand{\OperatorTok}[1]{\textcolor[rgb]{0.81,0.36,0.00}{\textbf{#1}}}
\newcommand{\OtherTok}[1]{\textcolor[rgb]{0.56,0.35,0.01}{#1}}
\newcommand{\PreprocessorTok}[1]{\textcolor[rgb]{0.56,0.35,0.01}{\textit{#1}}}
\newcommand{\RegionMarkerTok}[1]{#1}
\newcommand{\SpecialCharTok}[1]{\textcolor[rgb]{0.00,0.00,0.00}{#1}}
\newcommand{\SpecialStringTok}[1]{\textcolor[rgb]{0.31,0.60,0.02}{#1}}
\newcommand{\StringTok}[1]{\textcolor[rgb]{0.31,0.60,0.02}{#1}}
\newcommand{\VariableTok}[1]{\textcolor[rgb]{0.00,0.00,0.00}{#1}}
\newcommand{\VerbatimStringTok}[1]{\textcolor[rgb]{0.31,0.60,0.02}{#1}}
\newcommand{\WarningTok}[1]{\textcolor[rgb]{0.56,0.35,0.01}{\textbf{\textit{#1}}}}
\usepackage{graphicx}
\makeatletter
\def\maxwidth{\ifdim\Gin@nat@width>\linewidth\linewidth\else\Gin@nat@width\fi}
\def\maxheight{\ifdim\Gin@nat@height>\textheight\textheight\else\Gin@nat@height\fi}
\makeatother
% Scale images if necessary, so that they will not overflow the page
% margins by default, and it is still possible to overwrite the defaults
% using explicit options in \includegraphics[width, height, ...]{}
\setkeys{Gin}{width=\maxwidth,height=\maxheight,keepaspectratio}
% Set default figure placement to htbp
\makeatletter
\def\fps@figure{htbp}
\makeatother
\setlength{\emergencystretch}{3em} % prevent overfull lines
\providecommand{\tightlist}{%
  \setlength{\itemsep}{0pt}\setlength{\parskip}{0pt}}
\setcounter{secnumdepth}{-\maxdimen} % remove section numbering
\ifLuaTeX
  \usepackage{selnolig}  % disable illegal ligatures
\fi

\begin{document}
\maketitle

\hypertarget{part-1-simulation-exercises}{%
\subsection{Part 1: Simulation
Exercises}\label{part-1-simulation-exercises}}

\textbf{Overview:}

Investigate the exponential distribution in R and compare it with the
Central Limit Theorem. The exponential distribution is simulated in R
with rexp(n, lambda) where lambda is the rate parameter. The mean of
exponential distribution is 1/lambda and the standard deviation is also
1/lambda. Set lambda = 0.2 for all of the simulations. Investigate the
distribution of averages of 40 exponentials. 1000 simulations are needed

\textbf{Questions:}

\begin{enumerate}
\def\labelenumi{\arabic{enumi}.}
\item
  Show the sample mean and compare it to the theoretical mean of the
  distribution.
\item
  Show how variable the sample is (via variance) and compare it to the
  theoretical variance of the distribution.
\item
  Show that the distribution is approximately normal.
\end{enumerate}

\begin{Shaded}
\begin{Highlighting}[]
\CommentTok{\# Set random seed}
\FunctionTok{set.seed}\NormalTok{(}\DecValTok{4569}\NormalTok{)}

\CommentTok{\# Set variables}
\NormalTok{lambda }\OtherTok{\textless{}{-}} \FloatTok{0.2}
\NormalTok{n }\OtherTok{\textless{}{-}} \DecValTok{40}
\NormalTok{sim }\OtherTok{\textless{}{-}} \DecValTok{1000}

\CommentTok{\# Run simulations}
\NormalTok{sim\_run }\OtherTok{\textless{}{-}} \FunctionTok{replicate}\NormalTok{(sim, }\FunctionTok{rexp}\NormalTok{(n, lambda))}

\CommentTok{\# Calculate mean of simulations}
\NormalTok{sim\_mean }\OtherTok{\textless{}{-}} \FunctionTok{apply}\NormalTok{(sim\_run, }\DecValTok{2}\NormalTok{, mean)}
\end{Highlighting}
\end{Shaded}

\textbf{Question 1: Theoretical Mean vs.~Sample Mean}

Theoretical mean:

\begin{Shaded}
\begin{Highlighting}[]
\NormalTok{mean\_t }\OtherTok{\textless{}{-}}\NormalTok{ lambda}\SpecialCharTok{\^{}{-}}\DecValTok{1}
\NormalTok{mean\_t}
\end{Highlighting}
\end{Shaded}

\begin{verbatim}
## [1] 5
\end{verbatim}

Sample mean:

\begin{Shaded}
\begin{Highlighting}[]
\NormalTok{mean\_sample }\OtherTok{\textless{}{-}} \FunctionTok{mean}\NormalTok{(sim\_mean)}
\NormalTok{mean\_sample}
\end{Highlighting}
\end{Shaded}

\begin{verbatim}
## [1] 5.008682
\end{verbatim}

Comparison:

Below is a histogram comparing the means. The line in red is the
theoretical mean, blue is sample mean

\begin{Shaded}
\begin{Highlighting}[]
\FunctionTok{hist}\NormalTok{(sim\_mean, }\AttributeTok{main =} \StringTok{"Mean Comparison"}\NormalTok{, }\AttributeTok{breaks=}\DecValTok{50}\NormalTok{)}
\FunctionTok{abline}\NormalTok{(}\AttributeTok{v =}\NormalTok{ mean\_t, }\AttributeTok{col =} \StringTok{"red"}\NormalTok{)}
\FunctionTok{abline}\NormalTok{(}\AttributeTok{v =}\NormalTok{ mean\_sample, }\AttributeTok{col =} \StringTok{"blue"}\NormalTok{)}
\end{Highlighting}
\end{Shaded}

\includegraphics{courseproject_files/figure-latex/unnamed-chunk-4-1.pdf}
From the graph, we can see the sample mean of 5.0086 is very close to
the theoretcial mean

\textbf{Question 2: Theoretical Variance vs.~Sample Variance}

For theoretical variance, the formula is (lambda*sqrt(n))\^{}-2:

\begin{Shaded}
\begin{Highlighting}[]
\NormalTok{var\_t }\OtherTok{\textless{}{-}}\NormalTok{ (lambda}\SpecialCharTok{*}\FunctionTok{sqrt}\NormalTok{(n))}\SpecialCharTok{\^{}{-}}\DecValTok{2}
\NormalTok{var\_t}
\end{Highlighting}
\end{Shaded}

\begin{verbatim}
## [1] 0.625
\end{verbatim}

Sample variance:

\begin{Shaded}
\begin{Highlighting}[]
\NormalTok{var\_sample }\OtherTok{\textless{}{-}} \FunctionTok{var}\NormalTok{(sim\_mean)}
\NormalTok{var\_sample}
\end{Highlighting}
\end{Shaded}

\begin{verbatim}
## [1] 0.6281728
\end{verbatim}

Comparing the two, we can see the theoretical and sample variances are
very close

\textbf{Question 3: Distribution}

This last question will show if the distribution is approximately normal

\begin{Shaded}
\begin{Highlighting}[]
\FunctionTok{hist}\NormalTok{(sim\_mean, }\AttributeTok{main =} \StringTok{"Normal Distribution"}\NormalTok{, }\AttributeTok{breaks=}\DecValTok{50}\NormalTok{)}

\NormalTok{x }\OtherTok{\textless{}{-}} \FunctionTok{seq}\NormalTok{(}\FunctionTok{min}\NormalTok{(sim\_mean), }\FunctionTok{max}\NormalTok{(sim\_mean), }\AttributeTok{length=}\DecValTok{100}\NormalTok{)}
\NormalTok{y }\OtherTok{\textless{}{-}} \FunctionTok{dnorm}\NormalTok{(x, }\AttributeTok{mean=}\DecValTok{1}\SpecialCharTok{/}\NormalTok{lambda, }\AttributeTok{sd=}\NormalTok{(}\DecValTok{1}\SpecialCharTok{/}\NormalTok{lambda)}\SpecialCharTok{/}\FunctionTok{sqrt}\NormalTok{(n))}
\FunctionTok{lines}\NormalTok{(x, y}\SpecialCharTok{*}\DecValTok{100}\NormalTok{, }\AttributeTok{col=}\StringTok{"blue"}\NormalTok{)}
\end{Highlighting}
\end{Shaded}

\includegraphics{courseproject_files/figure-latex/unnamed-chunk-7-1.pdf}

From the graph, we can see that the exponential distribution appears to
be normal centered on our mean.

\end{document}
